\documentclass[11pt]{article}

    \usepackage[breakable]{tcolorbox}
    \usepackage{parskip} % Stop auto-indenting (to mimic markdown behaviour)
    
    \usepackage{iftex}
    \ifPDFTeX
    	\usepackage[T1]{fontenc}
    	\usepackage{mathpazo}
    \else
    	\usepackage{fontspec}
    \fi

    % Basic figure setup, for now with no caption control since it's done
    % automatically by Pandoc (which extracts ![](path) syntax from Markdown).
    \usepackage{graphicx}
    % Maintain compatibility with old templates. Remove in nbconvert 6.0
    \let\Oldincludegraphics\includegraphics
    % Ensure that by default, figures have no caption (until we provide a
    % proper Figure object with a Caption API and a way to capture that
    % in the conversion process - todo).
    \usepackage{caption}
    \DeclareCaptionFormat{nocaption}{}
    \captionsetup{format=nocaption,aboveskip=0pt,belowskip=0pt}

    \usepackage[Export]{adjustbox} % Used to constrain images to a maximum size
    \adjustboxset{max size={0.9\linewidth}{0.9\paperheight}}
    \usepackage{float}
    \floatplacement{figure}{H} % forces figures to be placed at the correct location
    \usepackage{xcolor} % Allow colors to be defined
    \usepackage{enumerate} % Needed for markdown enumerations to work
    \usepackage{geometry} % Used to adjust the document margins
    \usepackage{amsmath} % Equations
    \usepackage{amssymb} % Equations
    \usepackage{textcomp} % defines textquotesingle
    % Hack from http://tex.stackexchange.com/a/47451/13684:
    \AtBeginDocument{%
        \def\PYZsq{\textquotesingle}% Upright quotes in Pygmentized code
    }
    \usepackage{upquote} % Upright quotes for verbatim code
    \usepackage{eurosym} % defines \euro
    \usepackage[mathletters]{ucs} % Extended unicode (utf-8) support
    \usepackage{fancyvrb} % verbatim replacement that allows latex
    \usepackage{grffile} % extends the file name processing of package graphics 
                         % to support a larger range
    \makeatletter % fix for grffile with XeLaTeX
    \def\Gread@@xetex#1{%
      \IfFileExists{"\Gin@base".bb}%
      {\Gread@eps{\Gin@base.bb}}%
      {\Gread@@xetex@aux#1}%
    }
    \makeatother

    % The hyperref package gives us a pdf with properly built
    % internal navigation ('pdf bookmarks' for the table of contents,
    % internal cross-reference links, web links for URLs, etc.)
    \usepackage{hyperref}
    % The default LaTeX title has an obnoxious amount of whitespace. By default,
    % titling removes some of it. It also provides customization options.
    \usepackage{titling}
    \usepackage{longtable} % longtable support required by pandoc >1.10
    \usepackage{booktabs}  % table support for pandoc > 1.12.2
    \usepackage[inline]{enumitem} % IRkernel/repr support (it uses the enumerate* environment)
    \usepackage[normalem]{ulem} % ulem is needed to support strikethroughs (\sout)
                                % normalem makes italics be italics, not underlines
    \usepackage{mathrsfs}
    

    
    % Colors for the hyperref package
    \definecolor{urlcolor}{rgb}{0,.145,.698}
    \definecolor{linkcolor}{rgb}{.71,0.21,0.01}
    \definecolor{citecolor}{rgb}{.12,.54,.11}

    % ANSI colors
    \definecolor{ansi-black}{HTML}{3E424D}
    \definecolor{ansi-black-intense}{HTML}{282C36}
    \definecolor{ansi-red}{HTML}{E75C58}
    \definecolor{ansi-red-intense}{HTML}{B22B31}
    \definecolor{ansi-green}{HTML}{00A250}
    \definecolor{ansi-green-intense}{HTML}{007427}
    \definecolor{ansi-yellow}{HTML}{DDB62B}
    \definecolor{ansi-yellow-intense}{HTML}{B27D12}
    \definecolor{ansi-blue}{HTML}{208FFB}
    \definecolor{ansi-blue-intense}{HTML}{0065CA}
    \definecolor{ansi-magenta}{HTML}{D160C4}
    \definecolor{ansi-magenta-intense}{HTML}{A03196}
    \definecolor{ansi-cyan}{HTML}{60C6C8}
    \definecolor{ansi-cyan-intense}{HTML}{258F8F}
    \definecolor{ansi-white}{HTML}{C5C1B4}
    \definecolor{ansi-white-intense}{HTML}{A1A6B2}
    \definecolor{ansi-default-inverse-fg}{HTML}{FFFFFF}
    \definecolor{ansi-default-inverse-bg}{HTML}{000000}

    % commands and environments needed by pandoc snippets
    % extracted from the output of `pandoc -s`
    \providecommand{\tightlist}{%
      \setlength{\itemsep}{0pt}\setlength{\parskip}{0pt}}
    \DefineVerbatimEnvironment{Highlighting}{Verbatim}{commandchars=\\\{\}}
    % Add ',fontsize=\small' for more characters per line
    \newenvironment{Shaded}{}{}
    \newcommand{\KeywordTok}[1]{\textcolor[rgb]{0.00,0.44,0.13}{\textbf{{#1}}}}
    \newcommand{\DataTypeTok}[1]{\textcolor[rgb]{0.56,0.13,0.00}{{#1}}}
    \newcommand{\DecValTok}[1]{\textcolor[rgb]{0.25,0.63,0.44}{{#1}}}
    \newcommand{\BaseNTok}[1]{\textcolor[rgb]{0.25,0.63,0.44}{{#1}}}
    \newcommand{\FloatTok}[1]{\textcolor[rgb]{0.25,0.63,0.44}{{#1}}}
    \newcommand{\CharTok}[1]{\textcolor[rgb]{0.25,0.44,0.63}{{#1}}}
    \newcommand{\StringTok}[1]{\textcolor[rgb]{0.25,0.44,0.63}{{#1}}}
    \newcommand{\CommentTok}[1]{\textcolor[rgb]{0.38,0.63,0.69}{\textit{{#1}}}}
    \newcommand{\OtherTok}[1]{\textcolor[rgb]{0.00,0.44,0.13}{{#1}}}
    \newcommand{\AlertTok}[1]{\textcolor[rgb]{1.00,0.00,0.00}{\textbf{{#1}}}}
    \newcommand{\FunctionTok}[1]{\textcolor[rgb]{0.02,0.16,0.49}{{#1}}}
    \newcommand{\RegionMarkerTok}[1]{{#1}}
    \newcommand{\ErrorTok}[1]{\textcolor[rgb]{1.00,0.00,0.00}{\textbf{{#1}}}}
    \newcommand{\NormalTok}[1]{{#1}}
    
    % Additional commands for more recent versions of Pandoc
    \newcommand{\ConstantTok}[1]{\textcolor[rgb]{0.53,0.00,0.00}{{#1}}}
    \newcommand{\SpecialCharTok}[1]{\textcolor[rgb]{0.25,0.44,0.63}{{#1}}}
    \newcommand{\VerbatimStringTok}[1]{\textcolor[rgb]{0.25,0.44,0.63}{{#1}}}
    \newcommand{\SpecialStringTok}[1]{\textcolor[rgb]{0.73,0.40,0.53}{{#1}}}
    \newcommand{\ImportTok}[1]{{#1}}
    \newcommand{\DocumentationTok}[1]{\textcolor[rgb]{0.73,0.13,0.13}{\textit{{#1}}}}
    \newcommand{\AnnotationTok}[1]{\textcolor[rgb]{0.38,0.63,0.69}{\textbf{\textit{{#1}}}}}
    \newcommand{\CommentVarTok}[1]{\textcolor[rgb]{0.38,0.63,0.69}{\textbf{\textit{{#1}}}}}
    \newcommand{\VariableTok}[1]{\textcolor[rgb]{0.10,0.09,0.49}{{#1}}}
    \newcommand{\ControlFlowTok}[1]{\textcolor[rgb]{0.00,0.44,0.13}{\textbf{{#1}}}}
    \newcommand{\OperatorTok}[1]{\textcolor[rgb]{0.40,0.40,0.40}{{#1}}}
    \newcommand{\BuiltInTok}[1]{{#1}}
    \newcommand{\ExtensionTok}[1]{{#1}}
    \newcommand{\PreprocessorTok}[1]{\textcolor[rgb]{0.74,0.48,0.00}{{#1}}}
    \newcommand{\AttributeTok}[1]{\textcolor[rgb]{0.49,0.56,0.16}{{#1}}}
    \newcommand{\InformationTok}[1]{\textcolor[rgb]{0.38,0.63,0.69}{\textbf{\textit{{#1}}}}}
    \newcommand{\WarningTok}[1]{\textcolor[rgb]{0.38,0.63,0.69}{\textbf{\textit{{#1}}}}}
    
    
    % Define a nice break command that doesn't care if a line doesn't already
    % exist.
    \def\br{\hspace*{\fill} \\* }
    % Math Jax compatibility definitions
    \def\gt{>}
    \def\lt{<}
    \let\Oldtex\TeX
    \let\Oldlatex\LaTeX
    \renewcommand{\TeX}{\textrm{\Oldtex}}
    \renewcommand{\LaTeX}{\textrm{\Oldlatex}}
    % Document parameters
    % Document title
    \title{RandomNumbers-FirstSteps}
    
    
    
    
    
% Pygments definitions
\makeatletter
\def\PY@reset{\let\PY@it=\relax \let\PY@bf=\relax%
    \let\PY@ul=\relax \let\PY@tc=\relax%
    \let\PY@bc=\relax \let\PY@ff=\relax}
\def\PY@tok#1{\csname PY@tok@#1\endcsname}
\def\PY@toks#1+{\ifx\relax#1\empty\else%
    \PY@tok{#1}\expandafter\PY@toks\fi}
\def\PY@do#1{\PY@bc{\PY@tc{\PY@ul{%
    \PY@it{\PY@bf{\PY@ff{#1}}}}}}}
\def\PY#1#2{\PY@reset\PY@toks#1+\relax+\PY@do{#2}}

\expandafter\def\csname PY@tok@w\endcsname{\def\PY@tc##1{\textcolor[rgb]{0.73,0.73,0.73}{##1}}}
\expandafter\def\csname PY@tok@c\endcsname{\let\PY@it=\textit\def\PY@tc##1{\textcolor[rgb]{0.25,0.50,0.50}{##1}}}
\expandafter\def\csname PY@tok@cp\endcsname{\def\PY@tc##1{\textcolor[rgb]{0.74,0.48,0.00}{##1}}}
\expandafter\def\csname PY@tok@k\endcsname{\let\PY@bf=\textbf\def\PY@tc##1{\textcolor[rgb]{0.00,0.50,0.00}{##1}}}
\expandafter\def\csname PY@tok@kp\endcsname{\def\PY@tc##1{\textcolor[rgb]{0.00,0.50,0.00}{##1}}}
\expandafter\def\csname PY@tok@kt\endcsname{\def\PY@tc##1{\textcolor[rgb]{0.69,0.00,0.25}{##1}}}
\expandafter\def\csname PY@tok@o\endcsname{\def\PY@tc##1{\textcolor[rgb]{0.40,0.40,0.40}{##1}}}
\expandafter\def\csname PY@tok@ow\endcsname{\let\PY@bf=\textbf\def\PY@tc##1{\textcolor[rgb]{0.67,0.13,1.00}{##1}}}
\expandafter\def\csname PY@tok@nb\endcsname{\def\PY@tc##1{\textcolor[rgb]{0.00,0.50,0.00}{##1}}}
\expandafter\def\csname PY@tok@nf\endcsname{\def\PY@tc##1{\textcolor[rgb]{0.00,0.00,1.00}{##1}}}
\expandafter\def\csname PY@tok@nc\endcsname{\let\PY@bf=\textbf\def\PY@tc##1{\textcolor[rgb]{0.00,0.00,1.00}{##1}}}
\expandafter\def\csname PY@tok@nn\endcsname{\let\PY@bf=\textbf\def\PY@tc##1{\textcolor[rgb]{0.00,0.00,1.00}{##1}}}
\expandafter\def\csname PY@tok@ne\endcsname{\let\PY@bf=\textbf\def\PY@tc##1{\textcolor[rgb]{0.82,0.25,0.23}{##1}}}
\expandafter\def\csname PY@tok@nv\endcsname{\def\PY@tc##1{\textcolor[rgb]{0.10,0.09,0.49}{##1}}}
\expandafter\def\csname PY@tok@no\endcsname{\def\PY@tc##1{\textcolor[rgb]{0.53,0.00,0.00}{##1}}}
\expandafter\def\csname PY@tok@nl\endcsname{\def\PY@tc##1{\textcolor[rgb]{0.63,0.63,0.00}{##1}}}
\expandafter\def\csname PY@tok@ni\endcsname{\let\PY@bf=\textbf\def\PY@tc##1{\textcolor[rgb]{0.60,0.60,0.60}{##1}}}
\expandafter\def\csname PY@tok@na\endcsname{\def\PY@tc##1{\textcolor[rgb]{0.49,0.56,0.16}{##1}}}
\expandafter\def\csname PY@tok@nt\endcsname{\let\PY@bf=\textbf\def\PY@tc##1{\textcolor[rgb]{0.00,0.50,0.00}{##1}}}
\expandafter\def\csname PY@tok@nd\endcsname{\def\PY@tc##1{\textcolor[rgb]{0.67,0.13,1.00}{##1}}}
\expandafter\def\csname PY@tok@s\endcsname{\def\PY@tc##1{\textcolor[rgb]{0.73,0.13,0.13}{##1}}}
\expandafter\def\csname PY@tok@sd\endcsname{\let\PY@it=\textit\def\PY@tc##1{\textcolor[rgb]{0.73,0.13,0.13}{##1}}}
\expandafter\def\csname PY@tok@si\endcsname{\let\PY@bf=\textbf\def\PY@tc##1{\textcolor[rgb]{0.73,0.40,0.53}{##1}}}
\expandafter\def\csname PY@tok@se\endcsname{\let\PY@bf=\textbf\def\PY@tc##1{\textcolor[rgb]{0.73,0.40,0.13}{##1}}}
\expandafter\def\csname PY@tok@sr\endcsname{\def\PY@tc##1{\textcolor[rgb]{0.73,0.40,0.53}{##1}}}
\expandafter\def\csname PY@tok@ss\endcsname{\def\PY@tc##1{\textcolor[rgb]{0.10,0.09,0.49}{##1}}}
\expandafter\def\csname PY@tok@sx\endcsname{\def\PY@tc##1{\textcolor[rgb]{0.00,0.50,0.00}{##1}}}
\expandafter\def\csname PY@tok@m\endcsname{\def\PY@tc##1{\textcolor[rgb]{0.40,0.40,0.40}{##1}}}
\expandafter\def\csname PY@tok@gh\endcsname{\let\PY@bf=\textbf\def\PY@tc##1{\textcolor[rgb]{0.00,0.00,0.50}{##1}}}
\expandafter\def\csname PY@tok@gu\endcsname{\let\PY@bf=\textbf\def\PY@tc##1{\textcolor[rgb]{0.50,0.00,0.50}{##1}}}
\expandafter\def\csname PY@tok@gd\endcsname{\def\PY@tc##1{\textcolor[rgb]{0.63,0.00,0.00}{##1}}}
\expandafter\def\csname PY@tok@gi\endcsname{\def\PY@tc##1{\textcolor[rgb]{0.00,0.63,0.00}{##1}}}
\expandafter\def\csname PY@tok@gr\endcsname{\def\PY@tc##1{\textcolor[rgb]{1.00,0.00,0.00}{##1}}}
\expandafter\def\csname PY@tok@ge\endcsname{\let\PY@it=\textit}
\expandafter\def\csname PY@tok@gs\endcsname{\let\PY@bf=\textbf}
\expandafter\def\csname PY@tok@gp\endcsname{\let\PY@bf=\textbf\def\PY@tc##1{\textcolor[rgb]{0.00,0.00,0.50}{##1}}}
\expandafter\def\csname PY@tok@go\endcsname{\def\PY@tc##1{\textcolor[rgb]{0.53,0.53,0.53}{##1}}}
\expandafter\def\csname PY@tok@gt\endcsname{\def\PY@tc##1{\textcolor[rgb]{0.00,0.27,0.87}{##1}}}
\expandafter\def\csname PY@tok@err\endcsname{\def\PY@bc##1{\setlength{\fboxsep}{0pt}\fcolorbox[rgb]{1.00,0.00,0.00}{1,1,1}{\strut ##1}}}
\expandafter\def\csname PY@tok@kc\endcsname{\let\PY@bf=\textbf\def\PY@tc##1{\textcolor[rgb]{0.00,0.50,0.00}{##1}}}
\expandafter\def\csname PY@tok@kd\endcsname{\let\PY@bf=\textbf\def\PY@tc##1{\textcolor[rgb]{0.00,0.50,0.00}{##1}}}
\expandafter\def\csname PY@tok@kn\endcsname{\let\PY@bf=\textbf\def\PY@tc##1{\textcolor[rgb]{0.00,0.50,0.00}{##1}}}
\expandafter\def\csname PY@tok@kr\endcsname{\let\PY@bf=\textbf\def\PY@tc##1{\textcolor[rgb]{0.00,0.50,0.00}{##1}}}
\expandafter\def\csname PY@tok@bp\endcsname{\def\PY@tc##1{\textcolor[rgb]{0.00,0.50,0.00}{##1}}}
\expandafter\def\csname PY@tok@fm\endcsname{\def\PY@tc##1{\textcolor[rgb]{0.00,0.00,1.00}{##1}}}
\expandafter\def\csname PY@tok@vc\endcsname{\def\PY@tc##1{\textcolor[rgb]{0.10,0.09,0.49}{##1}}}
\expandafter\def\csname PY@tok@vg\endcsname{\def\PY@tc##1{\textcolor[rgb]{0.10,0.09,0.49}{##1}}}
\expandafter\def\csname PY@tok@vi\endcsname{\def\PY@tc##1{\textcolor[rgb]{0.10,0.09,0.49}{##1}}}
\expandafter\def\csname PY@tok@vm\endcsname{\def\PY@tc##1{\textcolor[rgb]{0.10,0.09,0.49}{##1}}}
\expandafter\def\csname PY@tok@sa\endcsname{\def\PY@tc##1{\textcolor[rgb]{0.73,0.13,0.13}{##1}}}
\expandafter\def\csname PY@tok@sb\endcsname{\def\PY@tc##1{\textcolor[rgb]{0.73,0.13,0.13}{##1}}}
\expandafter\def\csname PY@tok@sc\endcsname{\def\PY@tc##1{\textcolor[rgb]{0.73,0.13,0.13}{##1}}}
\expandafter\def\csname PY@tok@dl\endcsname{\def\PY@tc##1{\textcolor[rgb]{0.73,0.13,0.13}{##1}}}
\expandafter\def\csname PY@tok@s2\endcsname{\def\PY@tc##1{\textcolor[rgb]{0.73,0.13,0.13}{##1}}}
\expandafter\def\csname PY@tok@sh\endcsname{\def\PY@tc##1{\textcolor[rgb]{0.73,0.13,0.13}{##1}}}
\expandafter\def\csname PY@tok@s1\endcsname{\def\PY@tc##1{\textcolor[rgb]{0.73,0.13,0.13}{##1}}}
\expandafter\def\csname PY@tok@mb\endcsname{\def\PY@tc##1{\textcolor[rgb]{0.40,0.40,0.40}{##1}}}
\expandafter\def\csname PY@tok@mf\endcsname{\def\PY@tc##1{\textcolor[rgb]{0.40,0.40,0.40}{##1}}}
\expandafter\def\csname PY@tok@mh\endcsname{\def\PY@tc##1{\textcolor[rgb]{0.40,0.40,0.40}{##1}}}
\expandafter\def\csname PY@tok@mi\endcsname{\def\PY@tc##1{\textcolor[rgb]{0.40,0.40,0.40}{##1}}}
\expandafter\def\csname PY@tok@il\endcsname{\def\PY@tc##1{\textcolor[rgb]{0.40,0.40,0.40}{##1}}}
\expandafter\def\csname PY@tok@mo\endcsname{\def\PY@tc##1{\textcolor[rgb]{0.40,0.40,0.40}{##1}}}
\expandafter\def\csname PY@tok@ch\endcsname{\let\PY@it=\textit\def\PY@tc##1{\textcolor[rgb]{0.25,0.50,0.50}{##1}}}
\expandafter\def\csname PY@tok@cm\endcsname{\let\PY@it=\textit\def\PY@tc##1{\textcolor[rgb]{0.25,0.50,0.50}{##1}}}
\expandafter\def\csname PY@tok@cpf\endcsname{\let\PY@it=\textit\def\PY@tc##1{\textcolor[rgb]{0.25,0.50,0.50}{##1}}}
\expandafter\def\csname PY@tok@c1\endcsname{\let\PY@it=\textit\def\PY@tc##1{\textcolor[rgb]{0.25,0.50,0.50}{##1}}}
\expandafter\def\csname PY@tok@cs\endcsname{\let\PY@it=\textit\def\PY@tc##1{\textcolor[rgb]{0.25,0.50,0.50}{##1}}}

\def\PYZbs{\char`\\}
\def\PYZus{\char`\_}
\def\PYZob{\char`\{}
\def\PYZcb{\char`\}}
\def\PYZca{\char`\^}
\def\PYZam{\char`\&}
\def\PYZlt{\char`\<}
\def\PYZgt{\char`\>}
\def\PYZsh{\char`\#}
\def\PYZpc{\char`\%}
\def\PYZdl{\char`\$}
\def\PYZhy{\char`\-}
\def\PYZsq{\char`\'}
\def\PYZdq{\char`\"}
\def\PYZti{\char`\~}
% for compatibility with earlier versions
\def\PYZat{@}
\def\PYZlb{[}
\def\PYZrb{]}
\makeatother


    % For linebreaks inside Verbatim environment from package fancyvrb. 
    \makeatletter
        \newbox\Wrappedcontinuationbox 
        \newbox\Wrappedvisiblespacebox 
        \newcommand*\Wrappedvisiblespace {\textcolor{red}{\textvisiblespace}} 
        \newcommand*\Wrappedcontinuationsymbol {\textcolor{red}{\llap{\tiny$\m@th\hookrightarrow$}}} 
        \newcommand*\Wrappedcontinuationindent {3ex } 
        \newcommand*\Wrappedafterbreak {\kern\Wrappedcontinuationindent\copy\Wrappedcontinuationbox} 
        % Take advantage of the already applied Pygments mark-up to insert 
        % potential linebreaks for TeX processing. 
        %        {, <, #, %, $, ' and ": go to next line. 
        %        _, }, ^, &, >, - and ~: stay at end of broken line. 
        % Use of \textquotesingle for straight quote. 
        \newcommand*\Wrappedbreaksatspecials {% 
            \def\PYGZus{\discretionary{\char`\_}{\Wrappedafterbreak}{\char`\_}}% 
            \def\PYGZob{\discretionary{}{\Wrappedafterbreak\char`\{}{\char`\{}}% 
            \def\PYGZcb{\discretionary{\char`\}}{\Wrappedafterbreak}{\char`\}}}% 
            \def\PYGZca{\discretionary{\char`\^}{\Wrappedafterbreak}{\char`\^}}% 
            \def\PYGZam{\discretionary{\char`\&}{\Wrappedafterbreak}{\char`\&}}% 
            \def\PYGZlt{\discretionary{}{\Wrappedafterbreak\char`\<}{\char`\<}}% 
            \def\PYGZgt{\discretionary{\char`\>}{\Wrappedafterbreak}{\char`\>}}% 
            \def\PYGZsh{\discretionary{}{\Wrappedafterbreak\char`\#}{\char`\#}}% 
            \def\PYGZpc{\discretionary{}{\Wrappedafterbreak\char`\%}{\char`\%}}% 
            \def\PYGZdl{\discretionary{}{\Wrappedafterbreak\char`\$}{\char`\$}}% 
            \def\PYGZhy{\discretionary{\char`\-}{\Wrappedafterbreak}{\char`\-}}% 
            \def\PYGZsq{\discretionary{}{\Wrappedafterbreak\textquotesingle}{\textquotesingle}}% 
            \def\PYGZdq{\discretionary{}{\Wrappedafterbreak\char`\"}{\char`\"}}% 
            \def\PYGZti{\discretionary{\char`\~}{\Wrappedafterbreak}{\char`\~}}% 
        } 
        % Some characters . , ; ? ! / are not pygmentized. 
        % This macro makes them "active" and they will insert potential linebreaks 
        \newcommand*\Wrappedbreaksatpunct {% 
            \lccode`\~`\.\lowercase{\def~}{\discretionary{\hbox{\char`\.}}{\Wrappedafterbreak}{\hbox{\char`\.}}}% 
            \lccode`\~`\,\lowercase{\def~}{\discretionary{\hbox{\char`\,}}{\Wrappedafterbreak}{\hbox{\char`\,}}}% 
            \lccode`\~`\;\lowercase{\def~}{\discretionary{\hbox{\char`\;}}{\Wrappedafterbreak}{\hbox{\char`\;}}}% 
            \lccode`\~`\:\lowercase{\def~}{\discretionary{\hbox{\char`\:}}{\Wrappedafterbreak}{\hbox{\char`\:}}}% 
            \lccode`\~`\?\lowercase{\def~}{\discretionary{\hbox{\char`\?}}{\Wrappedafterbreak}{\hbox{\char`\?}}}% 
            \lccode`\~`\!\lowercase{\def~}{\discretionary{\hbox{\char`\!}}{\Wrappedafterbreak}{\hbox{\char`\!}}}% 
            \lccode`\~`\/\lowercase{\def~}{\discretionary{\hbox{\char`\/}}{\Wrappedafterbreak}{\hbox{\char`\/}}}% 
            \catcode`\.\active
            \catcode`\,\active 
            \catcode`\;\active
            \catcode`\:\active
            \catcode`\?\active
            \catcode`\!\active
            \catcode`\/\active 
            \lccode`\~`\~ 	
        }
    \makeatother

    \let\OriginalVerbatim=\Verbatim
    \makeatletter
    \renewcommand{\Verbatim}[1][1]{%
        %\parskip\z@skip
        \sbox\Wrappedcontinuationbox {\Wrappedcontinuationsymbol}%
        \sbox\Wrappedvisiblespacebox {\FV@SetupFont\Wrappedvisiblespace}%
        \def\FancyVerbFormatLine ##1{\hsize\linewidth
            \vtop{\raggedright\hyphenpenalty\z@\exhyphenpenalty\z@
                \doublehyphendemerits\z@\finalhyphendemerits\z@
                \strut ##1\strut}%
        }%
        % If the linebreak is at a space, the latter will be displayed as visible
        % space at end of first line, and a continuation symbol starts next line.
        % Stretch/shrink are however usually zero for typewriter font.
        \def\FV@Space {%
            \nobreak\hskip\z@ plus\fontdimen3\font minus\fontdimen4\font
            \discretionary{\copy\Wrappedvisiblespacebox}{\Wrappedafterbreak}
            {\kern\fontdimen2\font}%
        }%
        
        % Allow breaks at special characters using \PYG... macros.
        \Wrappedbreaksatspecials
        % Breaks at punctuation characters . , ; ? ! and / need catcode=\active 	
        \OriginalVerbatim[#1,codes*=\Wrappedbreaksatpunct]%
    }
    \makeatother

    % Exact colors from NB
    \definecolor{incolor}{HTML}{303F9F}
    \definecolor{outcolor}{HTML}{D84315}
    \definecolor{cellborder}{HTML}{CFCFCF}
    \definecolor{cellbackground}{HTML}{F7F7F7}
    
    % prompt
    \makeatletter
    \newcommand{\boxspacing}{\kern\kvtcb@left@rule\kern\kvtcb@boxsep}
    \makeatother
    \newcommand{\prompt}[4]{
        \ttfamily\llap{{\color{#2}[#3]:\hspace{3pt}#4}}\vspace{-\baselineskip}
    }
    

    
    % Prevent overflowing lines due to hard-to-break entities
    \sloppy 
    % Setup hyperref package
    \hypersetup{
      breaklinks=true,  % so long urls are correctly broken across lines
      colorlinks=true,
      urlcolor=urlcolor,
      linkcolor=linkcolor,
      citecolor=citecolor,
      }
    % Slightly bigger margins than the latex defaults
    
    \geometry{verbose,tmargin=1in,bmargin=1in,lmargin=1in,rmargin=1in}
    
    

\begin{document}
    
    \maketitle
    
    

    
    \hypertarget{random-numbers}{%
\section{Random Numbers}\label{random-numbers}}

    In this notebook we will experiment with random numbers and
probabilities. We will use some functions from SciPy, an extension of
Python that provides many powerful tools to do science. To use these
functions coneniently, execute the folowing cell (i.e., select it and
press Shift+Return):

    \begin{tcolorbox}[breakable, size=fbox, boxrule=1pt, pad at break*=1mm,colback=cellbackground, colframe=cellborder]
\prompt{In}{incolor}{1}{\boxspacing}
\begin{Verbatim}[commandchars=\\\{\}]
\PY{o}{\PYZpc{}}\PY{k}{pylab} inline
\end{Verbatim}
\end{tcolorbox}

    \begin{Verbatim}[commandchars=\\\{\}]
Populating the interactive namespace from numpy and matplotlib
    \end{Verbatim}

    When evaluating this cell, you might have gotten a message ``Populating
the interactive namespace from numpy and matplotlib''. You can safely
ignore it. Now we are ready to start!

    \hypertarget{rolling-dice}{%
\subsubsection{Rolling Dice}\label{rolling-dice}}

    Rolling a fair die should result in one of 6 possible outcomes (1, 2, 3,
4, 5 or 6), each with equal probability of 1/6. SciPy provides a
function called ``randint'' that does something similar. You can read
about ``randint'', or most other functions, simply by evaluating a cell
with the function name, followed by a question mark. For example, when
you evaluate the following cell, a window will pop up to tell you about
the ``randint'' function (fell free to close it when you are done):

    \begin{tcolorbox}[breakable, size=fbox, boxrule=1pt, pad at break*=1mm,colback=cellbackground, colframe=cellborder]
\prompt{In}{incolor}{2}{\boxspacing}
\begin{Verbatim}[commandchars=\\\{\}]
randint\PY{o}{?}
\end{Verbatim}
\end{tcolorbox}

    
    \begin{verbatim}
Docstring:
randint(low, high=None, size=None, dtype=int)

Return random integers from `low` (inclusive) to `high` (exclusive).

Return random integers from the "discrete uniform" distribution of
the specified dtype in the "half-open" interval [`low`, `high`). If
`high` is None (the default), then results are from [0, `low`).

.. note::
    New code should use the ``integers`` method of a ``default_rng()``
    instance instead; see `random-quick-start`.

Parameters
----------
low : int or array-like of ints
    Lowest (signed) integers to be drawn from the distribution (unless
    ``high=None``, in which case this parameter is one above the
    *highest* such integer).
high : int or array-like of ints, optional
    If provided, one above the largest (signed) integer to be drawn
    from the distribution (see above for behavior if ``high=None``).
    If array-like, must contain integer values
size : int or tuple of ints, optional
    Output shape.  If the given shape is, e.g., ``(m, n, k)``, then
    ``m * n * k`` samples are drawn.  Default is None, in which case a
    single value is returned.
dtype : dtype, optional
    Desired dtype of the result. Byteorder must be native.
    The default value is int.

    .. versionadded:: 1.11.0

Returns
-------
out : int or ndarray of ints
    `size`-shaped array of random integers from the appropriate
    distribution, or a single such random int if `size` not provided.

See Also
--------
random_integers : similar to `randint`, only for the closed
    interval [`low`, `high`], and 1 is the lowest value if `high` is
    omitted.
Generator.integers: which should be used for new code.

Examples
--------
>>> np.random.randint(2, size=10)
array([1, 0, 0, 0, 1, 1, 0, 0, 1, 0]) # random
>>> np.random.randint(1, size=10)
array([0, 0, 0, 0, 0, 0, 0, 0, 0, 0])

Generate a 2 x 4 array of ints between 0 and 4, inclusive:

>>> np.random.randint(5, size=(2, 4))
array([[4, 0, 2, 1], # random
       [3, 2, 2, 0]])

Generate a 1 x 3 array with 3 different upper bounds

>>> np.random.randint(1, [3, 5, 10])
array([2, 2, 9]) # random

Generate a 1 by 3 array with 3 different lower bounds

>>> np.random.randint([1, 5, 7], 10)
array([9, 8, 7]) # random

Generate a 2 by 4 array using broadcasting with dtype of uint8

>>> np.random.randint([1, 3, 5, 7], [[10], [20]], dtype=np.uint8)
array([[ 8,  6,  9,  7], # random
       [ 1, 16,  9, 12]], dtype=uint8)
Type:      builtin_function_or_method

    \end{verbatim}

    
    From the documentation, it seems that ``randint(6)'' should result in a
random integer that is greater or equal than 0 and less than 6. Let's
try it:

    \begin{tcolorbox}[breakable, size=fbox, boxrule=1pt, pad at break*=1mm,colback=cellbackground, colframe=cellborder]
\prompt{In}{incolor}{7}{\boxspacing}
\begin{Verbatim}[commandchars=\\\{\}]
\PY{n}{randint}\PY{p}{(}\PY{l+m+mi}{6}\PY{p}{)}
\end{Verbatim}
\end{tcolorbox}

            \begin{tcolorbox}[breakable, size=fbox, boxrule=.5pt, pad at break*=1mm, opacityfill=0]
\prompt{Out}{outcolor}{7}{\boxspacing}
\begin{Verbatim}[commandchars=\\\{\}]
4
\end{Verbatim}
\end{tcolorbox}
        
    To emulate rolling a die we want a number that is greater or equal than
1 and less or equal than 6, which we can accomplish by simply adding one
to the result of randint:

    \begin{tcolorbox}[breakable, size=fbox, boxrule=1pt, pad at break*=1mm,colback=cellbackground, colframe=cellborder]
\prompt{In}{incolor}{ }{\boxspacing}
\begin{Verbatim}[commandchars=\\\{\}]
\PY{n}{randint}\PY{p}{(}\PY{l+m+mi}{6}\PY{p}{)}\PY{o}{+}\PY{l+m+mi}{1}
\end{Verbatim}
\end{tcolorbox}

    Now let's roll the die 100 times, i.e, let's create 100 independt random
numbers. We will save the result in a variable; let's call it x:

    \begin{tcolorbox}[breakable, size=fbox, boxrule=1pt, pad at break*=1mm,colback=cellbackground, colframe=cellborder]
\prompt{In}{incolor}{8}{\boxspacing}
\begin{Verbatim}[commandchars=\\\{\}]
\PY{n}{x}\PY{o}{=}\PY{n}{randint}\PY{p}{(}\PY{l+m+mi}{6}\PY{p}{,} \PY{n}{size}\PY{o}{=}\PY{l+m+mi}{100}\PY{p}{)}\PY{o}{+}\PY{l+m+mi}{1}
\PY{n+nb}{print} \PY{p}{(}\PY{n}{x}\PY{p}{)}
\end{Verbatim}
\end{tcolorbox}

    \begin{Verbatim}[commandchars=\\\{\}]
[6 3 6 3 2 4 2 6 1 3 1 1 5 3 2 6 1 3 1 1 4 2 3 2 1 5 3 3 6 3 3 2 3 5 4 4 3
 5 4 1 6 3 1 5 2 1 4 3 2 3 1 3 6 6 3 2 4 2 4 6 5 6 3 6 1 5 4 5 5 4 4 5 2 1
 4 1 5 1 1 5 6 5 3 5 5 6 2 6 6 2 6 6 4 5 3 2 1 6 1 3]
    \end{Verbatim}

    Statistically, each of the 6 possible outcomes should occur equally
often. To test this we can make a histogram of the 100 observations we
just made. You can use the ``hist'' function to do that quickly
(remember that you can execute the command ``hist?'' to learn about the
hist function):

    \begin{tcolorbox}[breakable, size=fbox, boxrule=1pt, pad at break*=1mm,colback=cellbackground, colframe=cellborder]
\prompt{In}{incolor}{9}{\boxspacing}
\begin{Verbatim}[commandchars=\\\{\}]
\PY{n}{hist}\PY{p}{(}\PY{n}{x}\PY{p}{,}\PY{n+nb}{range}\PY{o}{=}\PY{p}{(}\PY{l+m+mf}{0.5}\PY{p}{,}\PY{l+m+mf}{6.5}\PY{p}{)}\PY{p}{,}\PY{n}{bins}\PY{o}{=}\PY{l+m+mi}{6}\PY{p}{)}
\end{Verbatim}
\end{tcolorbox}

            \begin{tcolorbox}[breakable, size=fbox, boxrule=.5pt, pad at break*=1mm, opacityfill=0]
\prompt{Out}{outcolor}{9}{\boxspacing}
\begin{Verbatim}[commandchars=\\\{\}]
(array([18., 14., 21., 13., 16., 18.]),
 array([0.5, 1.5, 2.5, 3.5, 4.5, 5.5, 6.5]),
 <a list of 6 Patch objects>)
\end{Verbatim}
\end{tcolorbox}
        
    \begin{center}
    \adjustimage{max size={0.9\linewidth}{0.9\paperheight}}{RandomNumbers-FirstSteps_files/RandomNumbers-FirstSteps_14_1.png}
    \end{center}
    { \hspace*{\fill} \\}
    
    Beautiful! Notice that we created a histogram with 6 evenly spaced bins
between 0.5 and 6.5. Rather than producing a histogram, we can also use
the ``hist'' function to compute the probability distribution with the
additional keyword ``normed=True''. While we are at it, let's also
include some axis labels:

    \begin{tcolorbox}[breakable, size=fbox, boxrule=1pt, pad at break*=1mm,colback=cellbackground, colframe=cellborder]
\prompt{In}{incolor}{10}{\boxspacing}
\begin{Verbatim}[commandchars=\\\{\}]
\PY{n}{hist}\PY{p}{(}\PY{n}{x}\PY{p}{,}\PY{n+nb}{range}\PY{o}{=}\PY{p}{(}\PY{l+m+mf}{0.5}\PY{p}{,}\PY{l+m+mf}{6.5}\PY{p}{)}\PY{p}{,}\PY{n}{bins}\PY{o}{=}\PY{l+m+mi}{6}\PY{p}{,}\PY{n}{density}\PY{o}{=}\PY{k+kc}{True}\PY{p}{)}
\PY{n}{xlabel}\PY{p}{(}\PY{l+s+s2}{\PYZdq{}}\PY{l+s+s2}{Outcome}\PY{l+s+s2}{\PYZdq{}}\PY{p}{)}
\PY{n}{ylabel}\PY{p}{(}\PY{l+s+s2}{\PYZdq{}}\PY{l+s+s2}{Probability}\PY{l+s+s2}{\PYZdq{}}\PY{p}{)}
\end{Verbatim}
\end{tcolorbox}

            \begin{tcolorbox}[breakable, size=fbox, boxrule=.5pt, pad at break*=1mm, opacityfill=0]
\prompt{Out}{outcolor}{10}{\boxspacing}
\begin{Verbatim}[commandchars=\\\{\}]
Text(0, 0.5, 'Probability')
\end{Verbatim}
\end{tcolorbox}
        
    \begin{center}
    \adjustimage{max size={0.9\linewidth}{0.9\paperheight}}{RandomNumbers-FirstSteps_files/RandomNumbers-FirstSteps_16_1.png}
    \end{center}
    { \hspace*{\fill} \\}
    
    You see that the probability of each possible outcome is somewhere close
to 1/6. Let's do this again, but this time we roll the die 100000 times
(don't try to do that with a real die):

    \begin{tcolorbox}[breakable, size=fbox, boxrule=1pt, pad at break*=1mm,colback=cellbackground, colframe=cellborder]
\prompt{In}{incolor}{11}{\boxspacing}
\begin{Verbatim}[commandchars=\\\{\}]
\PY{n}{x}\PY{o}{=}\PY{n}{randint}\PY{p}{(}\PY{l+m+mi}{6}\PY{p}{,} \PY{n}{size}\PY{o}{=}\PY{l+m+mi}{100000}\PY{p}{)}\PY{o}{+}\PY{l+m+mi}{1}
\PY{n}{hist}\PY{p}{(}\PY{n}{x}\PY{p}{,}\PY{n+nb}{range}\PY{o}{=}\PY{p}{(}\PY{l+m+mf}{0.5}\PY{p}{,}\PY{l+m+mf}{6.5}\PY{p}{)}\PY{p}{,}\PY{n}{bins}\PY{o}{=}\PY{l+m+mi}{6}\PY{p}{,}\PY{n}{density}\PY{o}{=}\PY{k+kc}{True}\PY{p}{)}
\PY{n}{xlabel}\PY{p}{(}\PY{l+s+s2}{\PYZdq{}}\PY{l+s+s2}{Outcome}\PY{l+s+s2}{\PYZdq{}}\PY{p}{)}
\PY{n}{ylabel}\PY{p}{(}\PY{l+s+s2}{\PYZdq{}}\PY{l+s+s2}{Probability}\PY{l+s+s2}{\PYZdq{}}\PY{p}{)}
\end{Verbatim}
\end{tcolorbox}

            \begin{tcolorbox}[breakable, size=fbox, boxrule=.5pt, pad at break*=1mm, opacityfill=0]
\prompt{Out}{outcolor}{11}{\boxspacing}
\begin{Verbatim}[commandchars=\\\{\}]
Text(0, 0.5, 'Probability')
\end{Verbatim}
\end{tcolorbox}
        
    \begin{center}
    \adjustimage{max size={0.9\linewidth}{0.9\paperheight}}{RandomNumbers-FirstSteps_files/RandomNumbers-FirstSteps_18_1.png}
    \end{center}
    { \hspace*{\fill} \\}
    
    You see that if you have many more samples, your observation will be
much closer to the true probability distribution.

    We can not only draw histograms, we can also quickly compute the mean of
our observations:

    \begin{tcolorbox}[breakable, size=fbox, boxrule=1pt, pad at break*=1mm,colback=cellbackground, colframe=cellborder]
\prompt{In}{incolor}{12}{\boxspacing}
\begin{Verbatim}[commandchars=\\\{\}]
\PY{n+nb}{print} \PY{p}{(}\PY{l+s+s2}{\PYZdq{}}\PY{l+s+s2}{Mean:}\PY{l+s+s2}{\PYZdq{}}\PY{p}{,} \PY{n}{mean}\PY{p}{(}\PY{n}{x}\PY{p}{)}\PY{p}{)}
\PY{n+nb}{print} \PY{p}{(}\PY{l+s+s2}{\PYZdq{}}\PY{l+s+s2}{Variance:}\PY{l+s+s2}{\PYZdq{}}\PY{p}{,} \PY{n}{var}\PY{p}{(}\PY{n}{x}\PY{p}{)}\PY{p}{)}
\end{Verbatim}
\end{tcolorbox}

    \begin{Verbatim}[commandchars=\\\{\}]
Mean: 3.49828
Variance: 2.9263570416
    \end{Verbatim}

    \hypertarget{continuous-random-numbers}{%
\subsubsection{Continuous Random
Numbers}\label{continuous-random-numbers}}

    In the die rolling example above we considered an expeirment that had
only 6 possible outcomes. Now let's look at a process with a continuous
sample space: picking a random number between 0 and 1 with uniform
probability.

    Creating such numbers can be easily done in Python; you can read about
it in the documentation of the ``rand'' function:

    \begin{tcolorbox}[breakable, size=fbox, boxrule=1pt, pad at break*=1mm,colback=cellbackground, colframe=cellborder]
\prompt{In}{incolor}{13}{\boxspacing}
\begin{Verbatim}[commandchars=\\\{\}]
rand\PY{o}{?}
\end{Verbatim}
\end{tcolorbox}

    
    \begin{verbatim}
Docstring:
rand(d0, d1, ..., dn)

Random values in a given shape.

.. note::
    This is a convenience function for users porting code from Matlab,
    and wraps `random_sample`. That function takes a
    tuple to specify the size of the output, which is consistent with
    other NumPy functions like `numpy.zeros` and `numpy.ones`.

Create an array of the given shape and populate it with
random samples from a uniform distribution
over ``[0, 1)``.

Parameters
----------
d0, d1, ..., dn : int, optional
    The dimensions of the returned array, must be non-negative.
    If no argument is given a single Python float is returned.

Returns
-------
out : ndarray, shape ``(d0, d1, ..., dn)``
    Random values.

See Also
--------
random

Examples
--------
>>> np.random.rand(3,2)
array([[ 0.14022471,  0.96360618],  #random
       [ 0.37601032,  0.25528411],  #random
       [ 0.49313049,  0.94909878]]) #random
Type:      builtin_function_or_method

    \end{verbatim}

    
    So, let's go ahead and generate 100 such random numbers:

    \begin{tcolorbox}[breakable, size=fbox, boxrule=1pt, pad at break*=1mm,colback=cellbackground, colframe=cellborder]
\prompt{In}{incolor}{14}{\boxspacing}
\begin{Verbatim}[commandchars=\\\{\}]
\PY{n}{x}\PY{o}{=}\PY{n}{rand}\PY{p}{(}\PY{l+m+mi}{100}\PY{p}{)}
\PY{n+nb}{print} \PY{p}{(}\PY{n}{x}\PY{p}{)}
\end{Verbatim}
\end{tcolorbox}

    \begin{Verbatim}[commandchars=\\\{\}]
[0.68815902 0.88637705 0.55708936 0.59854085 0.42875998 0.9111312
 0.98290583 0.06866527 0.56250665 0.14641319 0.35021375 0.54879893
 0.01755661 0.76870735 0.20569875 0.37130946 0.017682   0.82924424
 0.13778337 0.16855095 0.74632052 0.61351806 0.30909803 0.8265152
 0.25035231 0.51505246 0.78048977 0.72948129 0.09805457 0.65696482
 0.68976606 0.25523254 0.98770715 0.96944071 0.37416471 0.70391803
 0.94282506 0.74082752 0.94193627 0.9847373  0.08574497 0.25585229
 0.16843829 0.92967006 0.22109882 0.12322559 0.87856505 0.15698083
 0.31094478 0.46694811 0.80377383 0.25551756 0.70772748 0.68579701
 0.86068254 0.02930845 0.64895257 0.78978701 0.66761535 0.62799929
 0.73295794 0.19343026 0.79138302 0.9871116  0.00636273 0.95728485
 0.74624634 0.88149768 0.92085504 0.03363315 0.2054777  0.86076398
 0.82275925 0.9124652  0.07624698 0.31084641 0.21519738 0.48080884
 0.37112408 0.59495376 0.75480938 0.44378643 0.36938373 0.34273565
 0.67495658 0.43382423 0.06570394 0.74414588 0.55817903 0.58099578
 0.88197237 0.16664838 0.95671045 0.32797173 0.79431229 0.66877857
 0.28050916 0.58540629 0.77114526 0.54081696]
    \end{Verbatim}

    Here is the probability distribution we obtain from this dataset,
together with the mean and the variance:

    \begin{tcolorbox}[breakable, size=fbox, boxrule=1pt, pad at break*=1mm,colback=cellbackground, colframe=cellborder]
\prompt{In}{incolor}{15}{\boxspacing}
\begin{Verbatim}[commandchars=\\\{\}]
\PY{n}{hist}\PY{p}{(}\PY{n}{x}\PY{p}{,}\PY{n}{density}\PY{o}{=}\PY{k+kc}{True}\PY{p}{)}
\PY{n}{xlabel}\PY{p}{(}\PY{l+s+s2}{\PYZdq{}}\PY{l+s+s2}{x}\PY{l+s+s2}{\PYZdq{}}\PY{p}{)}
\PY{n}{ylabel}\PY{p}{(}\PY{l+s+s2}{\PYZdq{}}\PY{l+s+s2}{p(x)}\PY{l+s+s2}{\PYZdq{}}\PY{p}{)}
\PY{n+nb}{print} \PY{p}{(}\PY{l+s+s2}{\PYZdq{}}\PY{l+s+s2}{Mean:}\PY{l+s+s2}{\PYZdq{}}\PY{p}{,} \PY{n}{mean}\PY{p}{(}\PY{n}{x}\PY{p}{)}\PY{p}{)}
\PY{n+nb}{print} \PY{p}{(}\PY{l+s+s2}{\PYZdq{}}\PY{l+s+s2}{Variance:}\PY{l+s+s2}{\PYZdq{}}\PY{p}{,} \PY{n}{var}\PY{p}{(}\PY{n}{x}\PY{p}{)}\PY{p}{)}
\end{Verbatim}
\end{tcolorbox}

    \begin{Verbatim}[commandchars=\\\{\}]
Mean: 0.5408132437222097
Variance: 0.08918042369622439
    \end{Verbatim}

    \begin{center}
    \adjustimage{max size={0.9\linewidth}{0.9\paperheight}}{RandomNumbers-FirstSteps_files/RandomNumbers-FirstSteps_29_1.png}
    \end{center}
    { \hspace*{\fill} \\}
    
    Let's see how those numbers change when we use a larger sample size, for
example 10000 samples:

    \begin{tcolorbox}[breakable, size=fbox, boxrule=1pt, pad at break*=1mm,colback=cellbackground, colframe=cellborder]
\prompt{In}{incolor}{16}{\boxspacing}
\begin{Verbatim}[commandchars=\\\{\}]
\PY{n}{x}\PY{o}{=}\PY{n}{rand}\PY{p}{(}\PY{l+m+mi}{10000}\PY{p}{)}
\PY{n}{hist}\PY{p}{(}\PY{n}{x}\PY{p}{,}\PY{n}{density}\PY{o}{=}\PY{k+kc}{True}\PY{p}{)}
\PY{n}{xlabel}\PY{p}{(}\PY{l+s+s2}{\PYZdq{}}\PY{l+s+s2}{x}\PY{l+s+s2}{\PYZdq{}}\PY{p}{)}
\PY{n}{ylabel}\PY{p}{(}\PY{l+s+s2}{\PYZdq{}}\PY{l+s+s2}{p(x)}\PY{l+s+s2}{\PYZdq{}}\PY{p}{)}
\PY{n+nb}{print} \PY{p}{(}\PY{l+s+s2}{\PYZdq{}}\PY{l+s+s2}{Mean:}\PY{l+s+s2}{\PYZdq{}}\PY{p}{,} \PY{n}{mean}\PY{p}{(}\PY{n}{x}\PY{p}{)}\PY{p}{)}
\PY{n+nb}{print} \PY{p}{(}\PY{l+s+s2}{\PYZdq{}}\PY{l+s+s2}{Variance:}\PY{l+s+s2}{\PYZdq{}}\PY{p}{,} \PY{n}{var}\PY{p}{(}\PY{n}{x}\PY{p}{)}\PY{p}{)}
\end{Verbatim}
\end{tcolorbox}

    \begin{Verbatim}[commandchars=\\\{\}]
Mean: 0.4983180161870467
Variance: 0.08393729915318966
    \end{Verbatim}

    \begin{center}
    \adjustimage{max size={0.9\linewidth}{0.9\paperheight}}{RandomNumbers-FirstSteps_files/RandomNumbers-FirstSteps_31_1.png}
    \end{center}
    { \hspace*{\fill} \\}
    
    So far we have looked at the probability distribution of picking a
single number randomly between 0 and 1. Now we ask a different question:
What is the probability distribution of the \emph{average} of picking N
such numbers?

    We will pick N random numbers, calculate the average, and do this over
and over again until we have enough oberservations of this average to
make a histogram. We will start with an empty list (called
``averages''), and add to that list each observation of the average.
Let's start with a small number of N, let's say N=2:

    \begin{tcolorbox}[breakable, size=fbox, boxrule=1pt, pad at break*=1mm,colback=cellbackground, colframe=cellborder]
\prompt{In}{incolor}{17}{\boxspacing}
\begin{Verbatim}[commandchars=\\\{\}]
\PY{n}{N}\PY{o}{=}\PY{l+m+mi}{2}
\PY{n}{averages} \PY{o}{=} \PY{p}{[}\PY{p}{]}                  \PY{c+c1}{\PYZsh{} this creates an empty list with the name \PYZdq{}averages\PYZdq{}}
\PY{k}{for} \PY{n}{i} \PY{o+ow}{in} \PY{n+nb}{range}\PY{p}{(}\PY{l+m+mi}{10000}\PY{p}{)}\PY{p}{:}         \PY{c+c1}{\PYZsh{} let\PYZsq{}s do the following 100000 times:   }
    \PY{n}{x}\PY{o}{=}\PY{n}{rand}\PY{p}{(}\PY{n}{N}\PY{p}{)}                  \PY{c+c1}{\PYZsh{}     pick N independent random numbers from a uniform distribution over [0,1]}
    \PY{n}{averages}\PY{o}{.}\PY{n}{append} \PY{p}{(}\PY{n}{mean}\PY{p}{(}\PY{n}{x}\PY{p}{)}\PY{p}{)}  \PY{c+c1}{\PYZsh{}     calculate the average of those N numbers, and add it to the list}
\PY{n}{hist}\PY{p}{(}\PY{n}{averages}\PY{p}{,}\PY{n}{bins}\PY{o}{=}\PY{l+m+mi}{100}\PY{p}{,}\PY{n}{density}\PY{o}{=}\PY{k+kc}{True}\PY{p}{)}   \PY{c+c1}{\PYZsh{} compute and display the histogram distribution}
\PY{n}{xlabel}\PY{p}{(}\PY{l+s+s2}{\PYZdq{}}\PY{l+s+s2}{Observed Mean}\PY{l+s+s2}{\PYZdq{}}\PY{p}{)}
\PY{n}{ylabel}\PY{p}{(}\PY{l+s+s2}{\PYZdq{}}\PY{l+s+s2}{Probability Density}\PY{l+s+s2}{\PYZdq{}}\PY{p}{)}
\PY{n+nb}{print} \PY{p}{(}\PY{l+s+s2}{\PYZdq{}}\PY{l+s+s2}{Mean:}\PY{l+s+s2}{\PYZdq{}}\PY{p}{,} \PY{n}{mean}\PY{p}{(}\PY{n}{averages}\PY{p}{)}\PY{p}{)}
\PY{n+nb}{print} \PY{p}{(}\PY{l+s+s2}{\PYZdq{}}\PY{l+s+s2}{Variance:}\PY{l+s+s2}{\PYZdq{}}\PY{p}{,} \PY{n}{var}\PY{p}{(}\PY{n}{averages}\PY{p}{)}\PY{p}{)}
\end{Verbatim}
\end{tcolorbox}

    \begin{Verbatim}[commandchars=\\\{\}]
Mean: 0.5004026490634754
Variance: 0.04098892701634022
    \end{Verbatim}

    \begin{center}
    \adjustimage{max size={0.9\linewidth}{0.9\paperheight}}{RandomNumbers-FirstSteps_files/RandomNumbers-FirstSteps_34_1.png}
    \end{center}
    { \hspace*{\fill} \\}
    
    You see that the probability distribution for the average of 2 random
numbers is already very different from the distribution of a single
random number.

    Let us do this again, but this time we average over 10 random numbers:

    \begin{tcolorbox}[breakable, size=fbox, boxrule=1pt, pad at break*=1mm,colback=cellbackground, colframe=cellborder]
\prompt{In}{incolor}{18}{\boxspacing}
\begin{Verbatim}[commandchars=\\\{\}]
\PY{n}{N}\PY{o}{=}\PY{l+m+mi}{10}
\PY{n}{averages} \PY{o}{=} \PY{p}{[}\PY{p}{]}                  \PY{c+c1}{\PYZsh{} this creates an empty list with the name \PYZdq{}averages\PYZdq{}}
\PY{k}{for} \PY{n}{i} \PY{o+ow}{in} \PY{n+nb}{range}\PY{p}{(}\PY{l+m+mi}{100000}\PY{p}{)}\PY{p}{:}        \PY{c+c1}{\PYZsh{} let\PYZsq{}s do the following 100000 times:   }
    \PY{n}{x}\PY{o}{=}\PY{n}{rand}\PY{p}{(}\PY{n}{N}\PY{p}{)}                  \PY{c+c1}{\PYZsh{}     pick N independent random numbers from a uniform distribution over [0,1]}
    \PY{n}{averages}\PY{o}{.}\PY{n}{append} \PY{p}{(}\PY{n}{mean}\PY{p}{(}\PY{n}{x}\PY{p}{)}\PY{p}{)}  \PY{c+c1}{\PYZsh{}     calculate the average of those N numbers, and add it to the list}
\PY{n}{hist}\PY{p}{(}\PY{n}{averages}\PY{p}{,}\PY{n}{bins}\PY{o}{=}\PY{l+m+mi}{100}\PY{p}{,}\PY{n}{density}\PY{o}{=}\PY{k+kc}{True}\PY{p}{)}   \PY{c+c1}{\PYZsh{} compute and display the histogram distribution}
\PY{n}{xlabel}\PY{p}{(}\PY{l+s+s2}{\PYZdq{}}\PY{l+s+s2}{Observed Mean}\PY{l+s+s2}{\PYZdq{}}\PY{p}{)}
\PY{n}{ylabel}\PY{p}{(}\PY{l+s+s2}{\PYZdq{}}\PY{l+s+s2}{Probability Density}\PY{l+s+s2}{\PYZdq{}}\PY{p}{)}
\PY{n+nb}{print} \PY{p}{(}\PY{l+s+s2}{\PYZdq{}}\PY{l+s+s2}{Mean:}\PY{l+s+s2}{\PYZdq{}}\PY{p}{,} \PY{n}{mean}\PY{p}{(}\PY{n}{averages}\PY{p}{)}\PY{p}{)}
\PY{n+nb}{print} \PY{p}{(}\PY{l+s+s2}{\PYZdq{}}\PY{l+s+s2}{Variance:}\PY{l+s+s2}{\PYZdq{}}\PY{p}{,} \PY{n}{var}\PY{p}{(}\PY{n}{averages}\PY{p}{)}\PY{p}{)}
\end{Verbatim}
\end{tcolorbox}

    \begin{Verbatim}[commandchars=\\\{\}]
Mean: 0.5002356517926007
Variance: 0.008398043918019881
    \end{Verbatim}

    \begin{center}
    \adjustimage{max size={0.9\linewidth}{0.9\paperheight}}{RandomNumbers-FirstSteps_files/RandomNumbers-FirstSteps_37_1.png}
    \end{center}
    { \hspace*{\fill} \\}
    
    This already looks very Gaussian! Let's do this again, but this time
with N=40. Before you evaluate this cell, please stop and think about
how would expect the distribution to look like, and what you think the
mean and variance will be based on what you've just seen!

    \begin{tcolorbox}[breakable, size=fbox, boxrule=1pt, pad at break*=1mm,colback=cellbackground, colframe=cellborder]
\prompt{In}{incolor}{19}{\boxspacing}
\begin{Verbatim}[commandchars=\\\{\}]
\PY{n}{N}\PY{o}{=}\PY{l+m+mi}{40}
\PY{n}{averages} \PY{o}{=} \PY{p}{[}\PY{p}{]}                  \PY{c+c1}{\PYZsh{} this creates an empty list with the name \PYZdq{}averages\PYZdq{}}
\PY{k}{for} \PY{n}{i} \PY{o+ow}{in} \PY{n+nb}{range}\PY{p}{(}\PY{l+m+mi}{100000}\PY{p}{)}\PY{p}{:}        \PY{c+c1}{\PYZsh{} let\PYZsq{}s do the following 100000 times:   }
    \PY{n}{x}\PY{o}{=}\PY{n}{rand}\PY{p}{(}\PY{n}{N}\PY{p}{)}                  \PY{c+c1}{\PYZsh{}     pick N independent random numbers from a uniform distribution over [0,1]}
    \PY{n}{averages}\PY{o}{.}\PY{n}{append} \PY{p}{(}\PY{n}{mean}\PY{p}{(}\PY{n}{x}\PY{p}{)}\PY{p}{)}  \PY{c+c1}{\PYZsh{}     calculate the average of those N numbers, and add it to the list}
\PY{n}{hist}\PY{p}{(}\PY{n}{averages}\PY{p}{,}\PY{n}{bins}\PY{o}{=}\PY{l+m+mi}{100}\PY{p}{,}\PY{n}{density}\PY{o}{=}\PY{k+kc}{True}\PY{p}{)}   \PY{c+c1}{\PYZsh{} compute and display the histogram distribution}
\PY{n}{xlabel}\PY{p}{(}\PY{l+s+s2}{\PYZdq{}}\PY{l+s+s2}{Observed Mean}\PY{l+s+s2}{\PYZdq{}}\PY{p}{)}
\PY{n}{ylabel}\PY{p}{(}\PY{l+s+s2}{\PYZdq{}}\PY{l+s+s2}{Probability Density}\PY{l+s+s2}{\PYZdq{}}\PY{p}{)}
\PY{n+nb}{print} \PY{p}{(}\PY{l+s+s2}{\PYZdq{}}\PY{l+s+s2}{Mean:}\PY{l+s+s2}{\PYZdq{}}\PY{p}{,} \PY{n}{mean}\PY{p}{(}\PY{n}{averages}\PY{p}{)}\PY{p}{)}
\PY{n+nb}{print} \PY{p}{(}\PY{l+s+s2}{\PYZdq{}}\PY{l+s+s2}{Variance:}\PY{l+s+s2}{\PYZdq{}}\PY{p}{,} \PY{n}{var}\PY{p}{(}\PY{n}{averages}\PY{p}{)}\PY{p}{)}
\end{Verbatim}
\end{tcolorbox}

    \begin{Verbatim}[commandchars=\\\{\}]
Mean: 0.5000340114280194
Variance: 0.002093290537696063
    \end{Verbatim}

    \begin{center}
    \adjustimage{max size={0.9\linewidth}{0.9\paperheight}}{RandomNumbers-FirstSteps_files/RandomNumbers-FirstSteps_39_1.png}
    \end{center}
    { \hspace*{\fill} \\}
    
    The distribution is even more Gaussian, but it is now narrower. The mean
remains the same, but the variance should have decreased by a factor of
4 -- can you guess why?

    \begin{tcolorbox}[breakable, size=fbox, boxrule=1pt, pad at break*=1mm,colback=cellbackground, colframe=cellborder]
\prompt{In}{incolor}{22}{\boxspacing}
\begin{Verbatim}[commandchars=\\\{\}]
\PY{k}{def} \PY{n+nf}{gaussian}\PY{p}{(}\PY{n}{x}\PY{p}{,} \PY{n}{N}\PY{p}{,} \PY{n}{mean}\PY{p}{,} \PY{n}{variance}\PY{p}{)}\PY{p}{:}
    \PY{n}{coeff} \PY{o}{=} \PY{l+m+mi}{1}\PY{o}{/}\PY{n}{np}\PY{o}{.}\PY{n}{sqrt}\PY{p}{(}\PY{l+m+mi}{2}\PY{o}{*}\PY{n}{np}\PY{o}{.}\PY{n}{pi}\PY{o}{*}\PY{n}{variance}\PY{o}{/}\PY{n}{N}\PY{p}{)}
    \PY{n}{exponential} \PY{o}{=} \PY{o}{\PYZhy{}}\PY{p}{(}\PY{n}{x}\PY{o}{\PYZhy{}}\PY{n}{mean}\PY{p}{)}\PY{o}{*}\PY{o}{*}\PY{l+m+mi}{2}\PY{o}{/}\PY{p}{(}\PY{l+m+mi}{2}\PY{o}{*}\PY{n}{variance}\PY{o}{/}\PY{n}{N}\PY{p}{)}
    \PY{k}{return} \PY{n}{coeff}\PY{o}{*}\PY{n}{np}\PY{o}{.}\PY{n}{exp}\PY{p}{(}\PY{n}{exponential}\PY{p}{)}
\end{Verbatim}
\end{tcolorbox}

    \begin{tcolorbox}[breakable, size=fbox, boxrule=1pt, pad at break*=1mm,colback=cellbackground, colframe=cellborder]
\prompt{In}{incolor}{28}{\boxspacing}
\begin{Verbatim}[commandchars=\\\{\}]
\PY{n}{x\PYZus{}range} \PY{o}{=} \PY{n}{np}\PY{o}{.}\PY{n}{linspace}\PY{p}{(}\PY{l+m+mi}{0}\PY{p}{,}\PY{l+m+mi}{1}\PY{p}{,}\PY{l+m+mi}{101}\PY{p}{)}
\PY{n}{gaussian\PYZus{}x} \PY{o}{=} \PY{n}{gaussian}\PY{p}{(}\PY{n}{x\PYZus{}range}\PY{p}{,} \PY{l+m+mi}{100}\PY{p}{,} \PY{l+m+mf}{0.5}\PY{p}{,} \PY{p}{(}\PY{l+m+mi}{1}\PY{o}{/}\PY{l+m+mi}{12}\PY{p}{)}\PY{p}{)}
\end{Verbatim}
\end{tcolorbox}

    \begin{tcolorbox}[breakable, size=fbox, boxrule=1pt, pad at break*=1mm,colback=cellbackground, colframe=cellborder]
\prompt{In}{incolor}{30}{\boxspacing}
\begin{Verbatim}[commandchars=\\\{\}]
\PY{n}{N}\PY{o}{=}\PY{l+m+mi}{100}
\PY{n}{averages} \PY{o}{=} \PY{p}{[}\PY{p}{]}                  \PY{c+c1}{\PYZsh{} this creates an empty list with the name \PYZdq{}averages\PYZdq{}}
\PY{k}{for} \PY{n}{i} \PY{o+ow}{in} \PY{n+nb}{range}\PY{p}{(}\PY{l+m+mi}{100000}\PY{p}{)}\PY{p}{:}        \PY{c+c1}{\PYZsh{} let\PYZsq{}s do the following 100000 times:   }
    \PY{n}{x}\PY{o}{=}\PY{n}{rand}\PY{p}{(}\PY{n}{N}\PY{p}{)}                  \PY{c+c1}{\PYZsh{}     pick N independent random numbers from a uniform distribution over [0,1]}
    \PY{n}{averages}\PY{o}{.}\PY{n}{append} \PY{p}{(}\PY{n}{mean}\PY{p}{(}\PY{n}{x}\PY{p}{)}\PY{p}{)}  \PY{c+c1}{\PYZsh{}     calculate the average of those N numbers, and add it to the list}
\PY{n}{hist}\PY{p}{(}\PY{n}{averages}\PY{p}{,}\PY{n}{bins}\PY{o}{=}\PY{l+m+mi}{100}\PY{p}{,}\PY{n}{density}\PY{o}{=}\PY{k+kc}{True}\PY{p}{)}   \PY{c+c1}{\PYZsh{} compute and display the histogram distribution}
\PY{n}{plot}\PY{p}{(}\PY{n}{x\PYZus{}range}\PY{p}{,} \PY{n}{gaussian\PYZus{}x}\PY{p}{)}
\PY{n}{xlabel}\PY{p}{(}\PY{l+s+s2}{\PYZdq{}}\PY{l+s+s2}{Observed Mean}\PY{l+s+s2}{\PYZdq{}}\PY{p}{)}
\PY{n}{ylabel}\PY{p}{(}\PY{l+s+s2}{\PYZdq{}}\PY{l+s+s2}{Probability Density}\PY{l+s+s2}{\PYZdq{}}\PY{p}{)}
\PY{n+nb}{print} \PY{p}{(}\PY{l+s+s2}{\PYZdq{}}\PY{l+s+s2}{Mean:}\PY{l+s+s2}{\PYZdq{}}\PY{p}{,} \PY{n}{mean}\PY{p}{(}\PY{n}{averages}\PY{p}{)}\PY{p}{)}
\PY{n+nb}{print} \PY{p}{(}\PY{l+s+s2}{\PYZdq{}}\PY{l+s+s2}{Variance:}\PY{l+s+s2}{\PYZdq{}}\PY{p}{,} \PY{n}{var}\PY{p}{(}\PY{n}{averages}\PY{p}{)}\PY{p}{)}
\end{Verbatim}
\end{tcolorbox}

    \begin{Verbatim}[commandchars=\\\{\}]
Mean: 0.4999613444120334
Variance: 0.0008349441814745921
    \end{Verbatim}

    \begin{center}
    \adjustimage{max size={0.9\linewidth}{0.9\paperheight}}{RandomNumbers-FirstSteps_files/RandomNumbers-FirstSteps_43_1.png}
    \end{center}
    { \hspace*{\fill} \\}
    
    \begin{tcolorbox}[breakable, size=fbox, boxrule=1pt, pad at break*=1mm,colback=cellbackground, colframe=cellborder]
\prompt{In}{incolor}{ }{\boxspacing}
\begin{Verbatim}[commandchars=\\\{\}]

\end{Verbatim}
\end{tcolorbox}


    % Add a bibliography block to the postdoc
    
    
    
\end{document}
